
\documentclass[letterpaper,12pt,single]{article}
 \usepackage{color}  
% \usepackage{html}  
  \usepackage{times}  
  \usepackage{graphicx} 
 \usepackage{fancyheadings}  
%  \usepackage{hyperref}  
 \setlength{\parindent}{0.6pt} 
 \setlength{\parskip}{0.6pt} 
 \title{Music for Emma}
 \author{Mark}

\begin{document} 
\maketitle

\section{Blind Willie Johnson}

At first, the only thing there was was a crazy artist living in New York City named Harry Smith. 
He collected records of old-timey mountain folks from the south, singing songs that got passed down from one family to another. 
You saw ``Oh Brother, Where art thou?'' --- they were songs like that. 
In the 1950s, Harry put his favorite songs out on three records called The Anthology of American Folk Music. 
All of the coolest people in NYC and London started listening to those records.
They listened to them thousands of times until they wore out the needles on their record players. 
Those records had jug bands, and blues songs, murder ballads, and dudes who slid their knife up and down the guitar strings to make crazy sounds.
Nobody had ever heard anything like it before (at least not coming out of a record player.)
And then all those cool people in NYC and London started writing their own songs, imitating the styles they heard on the Anthology records. 
By that time it was the 19060s and the bands they were forming to make that music became the ``folk'' bands of the 1960s.
Blind Willie Johnson, was on the Anthology with this song, \textbf{John the Revelator}. 
A lot of the songs on the Anthology were about pretty nasty things, but Blind Willie Johnson was very religious. He sang about God, mostly.
Why was he called ``Blind'' Willie Johnson?
The story goes that he was blinded when his step-mom threw a pan of lye in his face when he was a kid. 
But very little is actually known about the guy.
Maybe it's better that way.
He was one of the most important people in the history of music.
And if we don't know much about him, then it is easy to imagine that it is because he had one foot on this planet, and another stood in some higher plane.
One of his songs is actually on board the Voyager spacecraft --- the first spacecraft to leave the solar system. (And, also, the evil alien force in Star Trek, The Motion Picture.)

\section{James Brown}

Have you listened to any James Brown yet? If you haven't you have a lot of catching up to do. His band was probably made up of the greatest musicians that ever lived. 
Nobody can play like those guys. 
And his music, sometimes called soul, sometimes called funk, set the basis for dance music for the next 50 years. 
He started putting these songs out in the 1960s, and people just couldn't not dance to them. 
This song, \textbf{Mother Popcorn} is about popcorn. And mothers, apparently.
The lyrics are funny, but listen to that music! 
Before James Brown dancing was boring. After him, it was fun. Turns out people like fun dancing a lot more than boring dancing.

\section{Leonard Cohen}

Lots of people (ELA teachers mostly) complain that people don't like poetry anymore. 
But this isn't true at all. 
What happened is that a bunch of the greatest poets of the 20th century just happened to be writing rock songs instead of ``poetry''. 
And nobody ever complains that people don't like rock songs, right? 
But some of the best rock song writers also just aren't very good singers (like Bob Dylan). 
Leonard Cohen is a Canadian dude who is one of the greatest song writers of all time. 
(He is also a pretty good, though maybe not great, singer.) 
He started writing songs in the 1960s. 
(He was probably listening to the Anthology of American Folk Music like everyone else.) 
He wrote that song ``Suzanne'' (\ldots``brings you tea and oranges that come all the way from China...'' --- you know it.) 
And he also wrote that song ``Allelujah'' --- which you would also recognize if you heard it. 
He wrote dozens of songs that will give you chills if you pay attention to them. 
Uncle Kevin says, "The man has class, heart, and cool." 
This song, \textbf{Famous Blue Raincoat}, is one of his best. 
It's better than Suzanne or Allelujah, even if more people know those ones.

\section{Serge Gainsbourg}

Serge Gainsbourg was a French man you never heard of. 
He was an ugly little guy who all the most beautiful women in the world were in love with because he wrote amazing songs. 
Those songs were in French though so most Americans have never heard of him. 
But the songs are really good, even if you don't speak French. 
The music is amazing and complex, but still catchy.
And if you listen to enough, somehow you can just tell that the lyrics are poetic and amazing.
Even though you don't understand them.
\textbf{The Ballad of Melody Nelson} is one song from a whole album that tells a story about a girl on a bicycle that he hits with his Rolls Royce, and then falls in love with. 
It is considered one of the greatest albums ever recorded (in French.)

\section{Johnny Cash}

Johnny Cash you know, because you saw that movie. 
But you should really listen to a whole lot more Johnny Cash. 
He did not write many of his own songs, and is not one of the greatest song writers of the 20th century. 
He's not a great singer either. 
Mostly he took other people's great songs and, through some kind of magic that I can't really figure out, he made them matter a lot more. 
Sometimes even when the original song was absolutely terrible. 
This song, \textbf{A Boy Named Sue} was written by Shel Silverstein. 
(You know, the guy who wrote ``Where the Sidewalk Ends'' --- same guy.) 
He played it for the first time in San Quentin Prison. 
That part is in the movie.

\section{Nancy Sinatra}

Nancy Sinatra is the daughter of Frank Sinatra (you know who he was right?) 
Most people think she sings pretty well. 
She sang more than one of the James Bond theme songs. 
And she sang that song ``These Boots Are Made For Walking.'' 
You've heard that song. 
But sort of like Johnny Cash, she didn't write most of her songs. 
(For a lot of performers back then, and even today, it's pretty normal to have someone else write your songs.)
She had a friend named Lee Hazlewood who wrote all of her most famous songs. 
He was one of the great song writers of all time. 
But he wasn't a very good singer, so he usually got Nancy to do that for him. 
This song, \textbf{Sand}, is one of their best, even if a lot more people have heard Boots.

\section{Kraftwerk}

Kraftwerk are a couple of German guys who are also the most influential band in the history of music. 
You've heard of the Beatles right? 
Kraftwerk is more important than they are. 
Why have you never heard of Kraftwerk? 
Because mostly it was other musicians who listened to them. 
A lot of other musicians, working in a far wider range of musical styles than the Beatles. 
What they did that is special is they made pop music with electronic instruments. 
Mostly keyboards. 
That sounds boring now, but nobody did it before them. 
And absolutely everyone did it after them because they thought Kraftwerk was so cool. 
Any band you have ever heard that has a keyboard in it, or a drum machine, or used clips from other music, was probably listening to Kraftwerk at some point. 
This includes ALL the pop music from the 1980s, ALL the dance music from the 1970s and 1980s (like, disco, maybe you've heard of that?) 
And even ALL of early rap and hip hop music (for some reason those guys in the Bronx who started rapping in the 1980s thought Kraftwerk was awesome.) 
But the thing is that Kraftwerk wasn't just breaking new ground with the instruments they used, their songs are amazing little puzzles that are always about two things at once. 
They are very simple melodies, with very simple lyrics, and very simple singing. 
But if you pay close attention you realize that all of them are little gems that you can listen to over and over again and get something new from them each time. 
Like any great music. 
This song, \textbf{Radioactivity} is both about radioactivity and about radios. With activity.

\section{Parliament}

Parliament or P-funk played a lot of their best stuff in the 1970s and they kind of picked up where James Brown left off. 
Like James Brown, they liked to cram as many people on stage as they could. 
And all of them were world-class musicians. 
They wrote crazy songs about aliens coming down from outer-space to make you dance. 
This was still called funk music because they weren't really using too many electronic instruments.
(Once there are lots of electronic instruments, people started calling it disco.) 
They had a real drummer and funky beats, without a drum machine. 
But their style was kind of like the crazy electronic dance music that was right around the corner (which did have drum machines) because everyone was listening to Kraftwerk.

\section{Brian Eno}

Also in the 1970s (see, living with Dad you had no idea the 1970s were so important to music, did you?) there was this guy Brian Eno. 
Brian Eno knew every musician in NYC and London. 
And they all wanted him to ``produce'' their albums. 
(That just means deciding which instruments to use on a song and how loud each instrument should be --- turns out this is really important to making a song sound good.) 
Eno was some kind of genius producer, as if God had reached down and tapped him on the head and said, ``thou shalt produce!''
He claimed not to know anything about music, but every album he touched magically turned exciting. 
He worked with Roxy Music, The Velvet Underground, U2, The Talking Heads, Genesis, David Bowie, Devo and piles of others. 
But, and this is a big but, he also wrote songs himself that have astonishing twisting lyrics that sometimes tells creepy stories, and sometimes make you fall in love. 
And his songs sound like nothing else before or since. 
He is one of the most important people in the history of music. This song, I'll Come Running, is one of his best.

\section{Donna Summer}

Donna Summer is famous for her disco songs. But what amazing disco songs they are! The lyrics aren't much, but her voice just soars. Her secret weapon though was this guy Giorgio Moroder. He was another producer with a gift for turning other people's songs into amazing artifacts. But he used more electronics and drum machines than Brian Eno. His sound was more like Kraftwerk. He turned this song, I Feel Love, by Donna Summer (and a bunch of others) into a work of art. When Brian Eno heard it, he said, "I have heard the sound of the future." At least, that's what David Bowie says he said. One way or another, this is the song that turned dance music from funk (with real instruments) into disco (with electronic instruments.)

\section{Patti Smith}

At the same time that Donna Summer was changing pop music, Patti Smith was changing punk rock. Before her, punk rock was just a bunch of dudes getting drunk and breaking things. She brought art and poetry and angry women to punk rock. Her songs are intense and full of imagery (like, images of say, horses.) And they do that awesome thing where they start off soft and build up until she's blowing the house down. Lots of the best bands do that. She is another person who knew every famous person who passed through NYC in the 1970s. Bruce Springsteen thought she was so cool he gave her one of his best songs! (Because The Night.) This song, Horses, is off her album, Horses, and is one of her most famous.

\section{Talking Heads}

Ok. There's another guy who was living in NYC in the 1970s and 80s named David Byrne. He had a band called The Talking Heads. They wrote pleasant pop songs that sounded good in the car when you were driving home from work. Turned out though that they lyrics were about serial killers and arson and the end of the world. Like Parliament and James Brown David Byrne had the good sense to pack the stage with unbelievable musicians, and you know, have Brian Eno produce his albums. He topped it off by having awesome designers and artists do his album covers. Every cool person in the world likes the Talking Heads. No joke. Go ahead and ask them. If you are ever at a party full of cool people and someone asks you to put some music on, just throw on some Talking Heads. Everyone will think you have good taste in music, but all you did was pick the one band that every cool person likes. This song, Nothing But Flowers, is about eating chocolate chip cookies after the apocalypse. It's my favorite. One of the backup singers on this song is Kirsty MacColl. She is important to the Pogues.

\section{Tom Waits}

Tom Waits sings with a gravelly low voice that is supposed to make you think he drank way too much whiskey and smoked far too many cigarettes. Maybe he did, maybe he didn't, but that's the schtick. Why does he want you to think that? Because he wants you to think his main job is singing for tips in a crappy bar. But don't be fooled, this guy is one of the finest muscians there is. And he packs his albums and shows full of world-class musicians. And then on top of that he lays down tricky, funny, touching, lyrics that are full of plays on words that Shakespeare would be jealous of. All that, and all you have to learn to like is a little bit of a gravelly voice. Really, he's not even as bad as Bob Dylan, and you know how much people will put up with to listen to \emph{his} songs.

\section{The Pogues}

The Pogues were an Irish rock band. They had this awesome idea that if you combined traditional irish music (you know with jigs and reels and whatnot) with punk and rock music, you would make a really really popular band. They were right. They were so right that nowadays there are about 1000 bands that sound like the Pogues. But the Pogues were there first, and they are a way better band than all the rest. Why? Because there's like nine dudes in the Pogues and every one of them is among the best in the world at playing his instrument. But lots of the other Irish rock bands have good musicians. The Pogues also had Shane MacGowan writing the songs for them. Songs just pour out of Shane MacGowan, it seemed like he didn't even have to try, he just threw words out there and they would somehow fit with the jigs and reels and rock in a way that would give you shivers. The only problem was he was so drunk all the time you couldn't understand a damn word he said. The band eventually kicked him out for being drunk. But then they had to let him back in, because they just weren't as good without him. People like well-played music, but they \emph{love} a catchy lyric and a good turn of phrase. This song, Fairtale of NY, is a duet with Kirsty MacColl (who sang with the Talking Heads, remember?) Her voice is shocking. Particularly on the lines, "you could have been someone - well so could anyone!" (The producer did a trick on that part where the vocals are doubled up so they sound like more than one person singing.) She later died saving her son from being run over by a motorboat. In Europe this song has actually become a Christmas classic, and people play it a lot during the holidays (probably an annoying amount.) Nick Cave says the first line is the greatest first line ever written to a song.

\section{Nirvana}

In the 1990s everyone was really miserable. Mostly because they had to wear really ugly clothes, and have bad haircuts, and nobody took showers. Plus, all the music and movies just weren't as good as they were in the 1970s and 1980s and nothing exciting really ever happened. So because of that, all the bands started singing about how miserable they all were, and nobody was more miserable than Kurt Cobain. So he had the most popular band of the time. He also had a gift for writing really good songs. The musicians? Kinda mediocre. But they are one of those bands that proves again that you can learn the bare minimum of actual music, but if you know how to twist a phrase in a catchy way, and set it to a good simple hook of music, people will worship the ground you walk on like a conquering king of old. Of course that didn't stop Kurt Cobain from killing himself.

\section{Nick Cave}

When Nick Cave was a kid, he liked to listen to Blind Willie Johnson, and Serge Gainsbourg, and Johnny Cash. But he \emph{loved} Leonard Cohen. He wanted to be Leonard Cohen, except he also wanted to do a lot of Heroin and sing really loud songs. So that's what he did. For decades he was a crazy person, he had a gang of filthy brutes with thick Austrlian accents that nobody could understand (and one German dude) and they wouldn't really play their instruments so much as beat the crap out of them. But when you could understand what Nick was singing, it was astonishing and beautiful dark lyrics which were scary, disturbing, gross, and romantic all at the same time. Sometime in the 1990s he stopped doing Heroin and his style mellowed out a lot, and suddenly everyone loved Nick Cave and his amazing lyrics.

\section{Will Oldham}

Will Oldham, also sometimes known as Bonnie 'Prince' Billie, is one more guy who has an uncanny knack for turning out songs that make your insides squishy. Sometime his lyrics are dark. Sometimes they are beautiful. But he can take a plain-seeming phrase and stick it to a bit of music that somehow just makes it bend into awesomeness. He has made lots and lots of recordings and they aren't all amazing (largely because he likes playing music a lot so sometimes he doesn't work as hard on the lyrics.) But he has written at least a dozen songs that are world-class, as good as anything ever written. This one, I See A Darkness, was so good that Johnny Cash himself covered it.


\end{document}
