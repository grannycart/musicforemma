% To compile this for html use:
% htlatex [this file name] marks-htlatex-css-config.cfg
% the .cfg file contains css lines to be incorporated into web site. It is what sets the font color, sans, and text width.
% It seems like it could be used to set just about anything that css can do. Cool!


\documentclass[letterpaper,single]{article}
% \usepackage{color}  % I included these in earlier versions, but I don't know why I thought I wanted them. 
% \usepackage{html}  
% \usepackage{times}  

 \usepackage{graphicx} 
% \usepackage{fancyhdr}  
 \usepackage{hyperref}  

\renewcommand{\familydefault}{\sfdefault} % Sans font for the pdf

% \renewcommand*{\familydefault}{\sfdefault} % Set sans serif as default font
% \setlength{\parindent}{0.6pt}  %not sure what these are for. Maybe necessary for tex files?
% \setlength{\parskip}{0.6pt} 
 \title{Music for Emma}
 \date{}
 \author{---}

\begin{document} 
\maketitle
\large
\itshape
This is a mix I made for my 15 year old sister Emma.
At some point in her life Emma will be at a party full of cool smart people, and someone will ask her to pick some music to play. I would hope (as my sister) she would have the confidence to spin some tunes that would make me proud and jam the party into the appropriate gear.
So the songs selected here are not obscure hidden gems, or challenging boundary pushers (not that there's anything wrong with boundary pushing), just some of my favorite tracks from some of the most important people in the history of pop music.
But since I don't think it is fair to just dump music on someone and expect them to try to figure out why it is good, so I wrote some brief descriptions to help fill in the story for her.\\
\\
You can download a pdf version of this \href{http://dynohub.net/musicforemma/texfiles/MusicforEmma.pdf}{\underline{here}}.\\
And you can download the playlist \href{http://dynohub.net/musicforemma/musicforemma-tracks.zip}{\underline{here}}.\\
\upshape


\section{Blind Willie Johnson}
In the beginning, before the Beatles, before Bob Dylan, and before Crosby Stills Nash and Young, there was a crazy artist living in New York City named Harry Smith. 
He collected records of old-timey mountain folks from the south, singing songs passed down from one family member to another for generations. 
You saw \emph{Oh Brother, Where art thou?} right?---that's what the songs were like. 
In the 1950s Harry made the most famous music mix of all time. 
He put all his favorite songs out on six records called \emph{The Anthology of American Folk Music}. 
All the cool people in NYC and London listened to those records. They listened to the Anthology until they wore out the needles of their record players. They couldn't get enough. 
Those records had jug bands, and blues songs; murder ballads, and dudes who slid their knife up and down the guitar strings to make crazy sounds.
Nobody had ever heard anything like it before. (At least not coming out of a record player.)
And then the cool people started writing their own songs, imitating the styles they heard on the Anthology records. 
By then it was the 1960s and the cool people were forming their own bands.
They wanted their bands to sound kind of like what they heard on the Anthology, and this became what you know as ``folk'' music.
Now, every song on the Anthology is a piece of musical history, and the Anthology songs influenced all sorts of people (including Bob Dylan who, as you know, is kind of a big deal.)
But Blind Willie is the hole in the middle of the Anthology around which the record of musical history spun.
In a whole collection of songs sounding like nothing else before, his song: \emph{John the Revelator}, demanded the particular attention of the cool people.
It sounded like it had come from some place deep in America's past, and at the same time it had a message for all of humanity's future.
And all the cool people felt bad because the guy who had written this song they loved so much had lived and died in a racist system, which meant none of them had even gotten to shake the fellow's hand.
He was kind of a mysterious figure. 
Not much is known about him except he was blind because his mom threw a pan of lye (a cleaning chemical) in his face when he was a kid---hence the name.
A lot of the songs on the Anthology were about nasty things, things just as nasty as getting lye thrown in your face, but Blind Willie Johnson was deeply religious. His songs were mostly about God.
It sometimes seemed like he stood with one foot planted on earth and the other on some higher plane.
Maybe he was a prophet.
But even if he was just a poor, blind, devout dude with a gravelly voice, he recorded some of the most important songs in the history of humanity.
One of his songs is actually on board the Voyager spacecraft---which became the first spacecraft to leave the solar system a couple of years ago. (And, also, is the evil alien force in Star Trek, The Motion Picture.)

\section{James Brown}
Have you listened to any James Brown yet? 
If you haven't, you have a lot of catching up to do. 
His band was made up of some of the greatest musicians who ever lived. 
Nobody can play like those guys. 
And the music they made, sometimes called soul, sometimes called funk, set the basis for dance music for the next 50 years. 
(Dance music---you know, music you dance to, as opposed to the kind you sit around and sing along with---is a pretty important part of music.) 
He started putting these songs out in the 1960s, and when people heard those songs, they just \emph{had} to start dancing. 
They couldn't help it. 
The lyrics could be ridiculous sometimes, but the music was an unstoppable dance force.
Before James Brown, dancing was kinda boring (there were a lot of rules for how to do it right). 
After him, it was fun. 
Turns out people like fun dancing a lot more than boring dancing.
This song, \textbf{Mother Popcorn} is about popcorn. And mothers, apparently.

\section{Leonard Cohen}
Lots of people (ELA teachers mostly) complain that most people don't like poetry anymore. 
But this isn't true at all. 
What actually happened is the greatest poets of the 20th century happened to prefer writing rock songs to writing ``poetry''. 
And nobody \emph{ever} complains that not enough people like rock songs, right? 
Leonard Cohen is a Canadian dude who is one of the greatest song writers of all time. 
(A lot of people think he is one of the greatest poets of all time too.) 
He started writing songs in the 1960s. 
(He was probably one of the cool people listening to the Anthology of American Folk Music like everyone else.) 
He wrote that song \emph{Suzanne} (``...brings you tea and oranges, that come all the way from China...''---you know the one.) 
And he also wrote a famous song called \emph{Allelujah} which you would recognize if you heard it. 
He wrote dozens of songs in which the way he constructs his lyrics will give you chills, if you give them a chance. 
Your Uncle Kevin says, ``The man has class, heart, and cool.''
This song, \textbf{Famous Blue Raincoat}, is one of his best.
One of the best songs ever written should be about a raincoat, don't you think?
% And It's better than \emph{Suzanne} or \emph{Allelujah}, even if more people know those songs.

\section{Serge Gainsbourg}
Serge Gainsbourg was a French man you never heard of. 
He was an ugly little troll who all the most beautiful women in the world were in love with because he wrote amazing songs. 
He too was a poet writing rock music.
But his songs were in French, so most Americans have never heard of him. 
The music is complex, but still catchy. (He liked to hire amazing musicians to play for him, just like James Brown.)
And if you listen to enough, somehow you can just tell the lyrics are deeply poetic and tricky, even though you don't understand them in French.
\textbf{The Ballad of Melody Nelson} is one song from a whole album about a girl riding a bicycle who he hits with his Rolls Royce---and then falls in love with. 
It is considered one of the greatest albums ever recorded (in French).

\section{Dolly Parton}
Dolly Parton is one of the few people included in this list who is both a world-class singer and a world-class songwriter at the same time.
Her singing is shocking---just listen to it---she has access to vocal registers that ``cut you to the quick'' as your old humbug wizard would say. 
Her voice just tears right through your sad little heart.
But the thing that makes Dolly special is she pairs her voice with the exceptionally good songs she writes.
Leonard Cohen probably writes better lyrics than Dolly Parton (well he pretty much writes better lyrics than \emph{everybody}) but he can't back his lyrics up with a voice like Dolly Parton's.
And when she puts those two things together it becomes something that makes you feel like it might spoil your reasonable sense of what other good music might sound like.
Dolly makes country music of course. And the best country songs are either really funny, or terribly sad, like this one: \textbf{I Will Always Love You}.
She has funny ones too.

\section{Johnny Cash}
Johnny Cash you know, because you saw that movie.
But you should listen to a lot \emph{more} Johnny Cash. 
He did not write many of his own songs, and is not one of the great song writers like other folks I've been talking about. 
Mostly he took other people's great songs and through some kind of magic I can't figure out, he made them matter a lot more.
Sometimes even when the original song was \emph{terrible}. 
My guess is his trick was using his stripped-down style to make other people's ridiculous and overly fancy songs, simple and sad (\ldots or funny!)
It didn't hurt that he had the mega-rock-star status thing going for him.
This song, \textbf{A Boy Named Sue} was written by Shel Silverstein. 
(You remember that book \emph{Where the Sidewalk Ends}?---same guy.) 
He played it for the first time in San Quentin prison. 
That part is in the movie.
The second most important dude in the history of country music is Merle Haggard. He was \emph{at} the show Johnny Cash played in San Quentin---because he was in prison.

\section{Nancy Sinatra}
Nancy Sinatra is the daughter of Frank Sinatra. (You know who he was, right?) 
Most people think she sings pretty well. (Though she's no Dolly Parton.)
She sang more than one James Bond theme song. 
And she sang that song \emph{These Boots Are Made For Walking}. 
I'm sure you've heard it. 
But like Johnny Cash, she didn't write most of her songs. 
(For a lot of performers back then, and even today, it's pretty normal to have someone else write your songs.)
She had a friend named Lee Hazlewood who wrote all of her most famous songs. 
And he \emph{was} one of the great song writers. 
But he wasn't a terrific singer, so he usually got Nancy to sing for him (smart, huh?). 
More people know \emph{These Boots Are Made For Walking}, but this song, \textbf{Sand}, is one of the best that Lee and Nancy ever did together.

\section{Kraftwerk}
Kraftwerk are a couple of German guys who also happen to be the most influential band in the history of music. 
You've heard of the Beatles right? 
Kraftwerk is more influential than they are. 
Then why have you never heard of Kraftwerk? 
Because mostly it was other musicians who listened to them. 
A \emph{lot} of other musicians. 
People who worked in a far wider range of musical styles than the Beatles. 
The special thing Kraftwerk did was they made pop music with electronic instruments. 
Mostly keyboards. 
That sounds boring now, but nobody did it before them. 
And \emph{everyone} did it after them because they thought Kraftwerk were so cool. 
Any band you have ever heard with a keyboard in it, or a drum machine, or uses clips from other music, probably was listening to Kraftwerk at some point. 
This includes \emph{all} the pop music from the 1980s, and \emph{all} the dance music from the 1970s and 1980s. (Like, say, disco---maybe you've heard of that?)
Remember, dance music is an enormous part of the history of music.
Even \emph{all} of the early rap and hip hop musicians were listening to Kraftwerk! 
For some reason the guys in the Bronx who started rapping in the 1980s (creating a whole new genre of music) thought this obscure band from Germany was awesome. 
But Kraftwerk wasn't just using new instruments to break new ground in musical history. 
Their songs were also amazing little puzzles which are always about two things at once. 
The songs have simple melodies, with simple lyrics, and simple singing. 
But if you pay close attention you'll realize the songs are special little things you can listen to over and over again, and hear a new meaning in the lyrics every time. 
Like any great music. 
This song, \textbf{Radioactivity} is both about radioactivity and about radios. With activity on them.

\section{Parliament}
Parliament (or P-funk) played a lot of their best stuff in the 1970s and they kind of picked up where James Brown left off. 
Like James Brown, they crammed as many people on stage as they could. 
And all of the people on stage were world-class musicians. 
They wrote crazy songs about aliens coming down from outer space to make you dance. 
Parliament's music was considered funk because they weren't using too many electronic instruments. 
(After them, when dance groups switched to using lots of electronic instruments, people started calling it disco.) 
They had a real drummer and real instruments and funky beats.
But their style was kind of like the crazy electronic dance music that was right around the corner. 
(Made by all those people listening to Kraftwerk, who thought computer music and keyboards were cool.)
The song here is \textbf{Flashlight}, which has the chorus, ``everybody has a little light under the sun.'' 
What do you think that means?

\section{Brian Eno}
Also in the 1970s (you had no idea the 1970s were so important to music, did you?) there was this guy Brian Eno. 
Brian Eno knew every musician in NYC and London. 
And they all wanted him to ``produce'' their albums. 
(That just means deciding which instruments to use on a song and how loud each instrument should be---turns out, this is important for making a song sound good.) 
Eno was some kind of genius producer, as if God had reached down and tapped him on the head and said, ``thou shalt produce!''
He claimed not to know anything about music, but every album he touched magically turned into a classic of pop music. 
He produced records for Roxy Music, The Velvet Underground, U2, Talking Heads, Genesis, David Bowie, Devo, and piles of others. 
But, and this is a big but, he also wrote songs himself which have astonishing twisting lyrics that sometimes tell creepy stories, and sometimes make you fall in love. 
And his songs sound like nothing else before or since. 
If you are talking about pop music, he is one of the most talented people to ever adjust the sound levels on a recording.
And this song, \textbf{I'll Come Running}, is one of the best he ever wrote himself.

\section{Donna Summer}
Donna Summer is famous for her disco songs. 
And what amazing disco songs they are! 
The lyrics aren't much, but her voice just soars. 
Her secret weapon though was a guy named Giorgio Moroder. 
He was another producer with a gift for turning other people's songs into pop hits. 
Not just ordinary pop hits---if you listen closely, the sounds he creates are complicated and have lots of smaller sounds mushed together to create them.
He used more electronics and computers than Brian Eno. 
His stuff sounded more like Kraftwerk. 
Many of the best songs of the disco era had Giorgio Moroder standing behind them, listening to headphones, and twiddling knobs until they sounded just perfect.
He turned this song, \textbf{I Feel Love}, by Donna Summer into a work of art. 
When Brian Eno heard it, he said, ``I have heard the sound of the future.'' 
At least, that's what David Bowie says he said. 
One way or another, this song turned dance music from funk (with real instruments) into disco (with electronic instruments.)

\section{Patti Smith}
At the same time that Donna Summer was changing pop music, Patti Smith was changing punk rock. 
Before her, punk rock was just a just bunch of dudes getting drunk and breaking things and yelling ``anarchy!'' (Not that there's anything wrong with that.) 
Patti Smith brought art and poetry and angry women to punk rock. 
Her songs are intense and full of imagery (like images of, say, horses.) 
And her band does the awesome thing where they start off soft and build up until you have to put your fingers in your ears and if you squint you can almost see raw energy pouring off the stage. 
Lots of the best bands do this kind of thing.
She is another person who knew every famous person who passed through NYC in the 1970s. 
Bruce Springsteen thought she was so cool he gave her one of his best songs! (\emph{Because The Night}.) This song, \textbf{Land}, is the centerpiece of her album \emph{Horses}---one of the most important albums in the history of music.

\section{Talking Heads}
Ok. Yet another guy living in NYC in the 1970s and 80s, named David Byrne, had a band called \emph{Talking Heads}. 
They wrote fun pop songs which sounded good in the car when you were driving home from work. 
But the lyrics were about serial killers, and arson, and the end of the world. 
Like Parliament and James Brown, David Byrne had the good sense to pack the stage with unbelievable musicians and (you may remember) have Brian Eno produce his albums. 
He topped it off by having awesome designers and artists do his album covers. 
Every cool person in the world likes Talking Heads. 
No joke. 
Go ahead and ask anyone cool you meet. 
If you are ever at a party full of cool people and someone asks you to put some music on, just throw on some Talking Heads. 
Everyone will think you have good taste in music, but all you did was pick the one band every cool person likes. 
This song, \textbf{Nothing But Flowers}, is about eating chocolate chip cookies after the apocalypse. It's my favorite. 
One of the backup singers on this song is Kirsty MacColl. She's part of the Pogues too, as you'll see.

\section{Tom Waits}
Tom Waits sings with a gravelly low voice which is supposed to make you think he drank too much whiskey and smoked too many cigarettes. 
Maybe he did, maybe he didn't, but that's the shtick. 
Why does he want you to think that? 
Because you're supposed to believe his main job is singing for tips in a crappy bar. 
But don't be fooled, this guy is one of the finest musicians there is. 
And he packs his albums and shows full of world-class backup musicians. 
And then on top of the music he lays down tricky, funny, touching, lyrics full of plays on words which make you giggle.
(They're ``punny''!)
All that, and all you have to do is learn to like a little bit of a gravelly voice. 
Really, his voice isn't even as bad as Bob Dylan's, and you know how much people will put up with to listen to \emph{that} guy's songs.
This song, \textbf{Strange Weather}, is one of the best songs nobody you know has ever heard.
It's about talking about the weather. And not buying umbrellas.

\section{The Pogues}
The Pogues were an Irish rock band. 
They had this awesome idea that if you combined traditional Irish music (you know jigs, reels, and whatnot) with punk and rock music, you could make a really \emph{really} popular band. 
They were right. 
They were so right that nowadays there are about 1000 bands which sound just like the Pogues. 
But the Pogues were there first, and they are a way better band than all the rest. 
Why? 
Because there's like nine dudes in the Pogues and every one of them is among the best in the world at playing his instrument. (Yet another band using that technique! Funny how well it works.) 
But lots of the other Irish rock bands have good musicians. 
The Pogues also had Shane MacGowan writing songs for them. 
Songs just poured out of Shane MacGowan. 
It seemed like he didn't even have to try. 
He just threw words out there and they would somehow fit the jig, and the reel, and the rock, in a way that gave you shivers. 
The only problem was he was so drunk all the time you couldn't understand a damn word he said. 
The band eventually kicked him out for being drunk. 
But then they had to let him back in, because without him they were just like all the other bands that sound like the Pogues.
People like well-played music, but they \emph{love} a catchy lyric and a good turn of phrase. 
This song, \textbf{Fairtale of NY}, is a duet with Kirsty MacColl (who sang with Talking Heads, remember?) 
Her voice is shocking. 
Particularly on the lines, ``you could have been someone---well so could anyone!''
(The producer did a trick on that part---the vocals are doubled up so they sound like more than one person singing.) 
She died young, saving her son from being run over by a motorboat. Which makes that song, and that part of the song particularly, kind of sad. But kind of sad in a good way.
In Europe this song has actually become a Christmas classic, and people play it a lot during the holidays (probably an annoying amount.) 
Nick Cave says the first line is the greatest first line ever written.
That's a pretty impressive endorsement, as you'll see.

\section{Nirvana}
In the 1990s everyone was miserable. 
Mostly because they had to wear ugly clothes that didn't fit, and have bad haircuts, and nobody took showers. 
Plus, all the music and movies just weren't as good as they were in the 1960s, '70s, and '80s. 
And nothing exciting ever happened. 
% (Seriously, one of the most famous papers of the time was called \emph{The End of History}. The guy who wrote it turned out to be wrong.)
Because of the miserable situation, all the bands started singing about how miserable they all were, and nobody was more miserable than Kurt Cobain. 
So of course that meant he had the most popular band of the time. 
He also had a gift for writing good songs. 
They are one of those bands proving once again you can learn the bare minimum of actual music, but if you know how to twist a phrase in a catchy way, and set it to a good simple hook of music, people will worship the ground you walk on like a conquering king of old. 
And the worship their king the people did, and they called their kingdom Grunge.
Of course that didn't stop Kurt Cobain from killing himself.
Most of Nirvana's songs were pretty noisy, but this song, \textbf{All Apologies}, probably captures what it felt like to live in the 1990s more than any other song written.

\section{Wu-Tang Clan}
In the 1990s popular rock music was getting boring. 
Everything kind of sounded the same, nobody was doing anything new.
At the same time, popular hip-hop was just launching a period which would eventually change the arc of music history affecting people right up until today.
In the 1960s folk music changed everything. In the 2000s hip-hop changed everything.
But before that happened, in the early 1990s, there was a battle going on in hip-hop between the East Coast (which mostly means NYC) and the West Coast (which mostly means Los Angeles.)
In 1993 the West Coast was winning. 
Their hip-hop was was serious and dark.
Nothing the East Coast made could stand up to it.
Until some dudes in NYC did what NYC always does---they got smart.
Enter the Wu-Tang Clan. The hip-hop band that eventually became Staten Island's greatest cultural export.
They put out an album in 1994 called \emph{Enter the Wu-Tang (36 Chambers)} which changed the way hip-hop worked.
Wu-Tang is yet another group with the great idea of cramming as many super-talented people on stage as they could fit.
Each member would take turns rapping a verse, and the verses were just as serious and dark as the Los Angeles stuff, but also incredibly intelligent, tricky, loaded with double meaning, and somehow funny at the same time.
Wu-Tang combined darkness with intelligence in a way that just wiped the slate of West Coast versus East Coast rap clean. It all started anew.
Every guy in Wu-Tang would go on to record their own albums as side-project, and every one of those albums would spawn a whole new sub-style of hip-hop.
Wu-Tang was like some kind of amazing chain reaction in the world of music. It just kept on going and going, changing everything it touched.
Sort of the exact opposite of the way popular rock was managing to completely not influence anyone at all in the 1990s.
This song \textbf{Protect Ya Neck} is the track that started it all.


\section{Nick Cave}
When Nick Cave was a kid, he liked to listen to Blind Willie Johnson, and Serge Gainsbourg, and Johnny Cash. 
But he \emph{loved} Leonard Cohen. 
He wanted to \emph{be} Leonard Cohen. 
Except he also wanted to do a lot of Heroin and sing loud songs. 
So that's what he did. 
For decades he was a crazy person. 
He had a gang of filthy brutes with thick Australian accents nobody could understand (and one German dude) and they wouldn't play their instruments so much as beat the crap out of them. 
But when you \emph{could} understand what Nick was singing, your found astonishing and beautiful dark lyrics which were scary, disturbing, gross, and romantic all at the same time. 
And oddly enough most of them were about God.
Sometime in the 1990s he stopped doing Heroin and his style mellowed out a lot, and suddenly (since you could now understand what he was singing) everyone loved Nick Cave and his amazing lyrics.
This song \textbf{I Let Love In} is from before that happened, but if you pay attention to the lyrics you can see why people like him so much.
There's only a handful of people in the history of music, maybe even entire the history of humanity, who can make a whole career of putting words to music in a way that makes the music transcend the words, and the words transcend the music at the same time---and the sum ends up greater than any of the parts.
(Most of them are somewhere on this list.)
But Nick Cave is my favorite of them all.

\section{Will Oldham}
Will Oldham, also sometimes known as Bonnie `Prince' Billie, is one more guy worth knowing about with an uncanny knack for turning out songs which make your insides squishy. 
Sometime his lyrics are dark. 
Sometimes they are beautiful. 
But he can take a plain-seeming phrase and stick it to a bit of music, and somehow it just bends into spine-tingly awesomeness. 
Kind of like if Dolly Parton were a grumpy guy with a big beard who sometimes sang about frightening things.
He has made lots and lots of recordings and they aren't all amazing.
(Largely because he likes playing music a lot, so maybe he doesn't always work as hard on the lyrics as he could.) 
But he has written at least a dozen world-class songs---as good as anything ever written. This one, \textbf{I See A Darkness}, was so good that Johnny Cash himself did a version of it.\\
\\
\\
\itshape
Since I originally wrote this list, I have realized that there were a few key entries that got left off.
Most notably Prince and Queen.
``But'' you say, ``what about Bob Dylan and David Bowie, surely they should be included, you even mention them a couple of times!''
They are not included for the simple reason that Emma was already familiar with them.
And I have to draw the line somewhere. Maybe someday I will expand this list, or make a part 2. 
But that won't happen until Emma takes a real interest in part 1 first.
\upshape
\\
\begin{center}
  \small
  \href{mailto://kunstcleaver@gmail.com}{Comments?}\\
  \textit{This work is licensed under a \\\href{http://creativecommons.org/licenses/by-sa/4.0/}{Creative Commons Attribution-ShareAlike 4.0 International License}}
  \textit{\today}\\
\end{center}
\end{document}
